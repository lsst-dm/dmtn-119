% DO NOT EDIT - generated by /Users/womullan/LSSTgit/lsst-texmf/bin/generateAcronyms.py from https://lsst-texmf.lsst.io/.
\newacronym{AD} {AD} {Associate Director}
\newglossaryentry{AURA} {name={AURA}, description={\gls{Association of Universities for Research in Astronomy}}}
\newglossaryentry{Archive} {name={Archive}, description={The repository for documents required by the NSF to be kept. These include documents related to design and development, construction, integration, test, and operations of the LSST observatory system. The archive is maintained using the enterprise content management system DocuShare, which is accessible through a link on the project website www.project.lsst.org.}}
\newglossaryentry{Association of Universities for Research in Astronomy} {name={Association of Universities for Research in Astronomy}, description={ consortium of US institutions and international affiliates that operates world-class astronomical observatories, AURA is the legal entity responsible for managing what it calls independent operating Centers, including LSST, under respective cooperative agreements with the National Science Foundation. AURA assumes fiducial responsibility for the funds provided through those cooperative agreements. AURA also is the legal owner of the AURA Observatory properties in Chile.}}
\newglossaryentry{CCD} {name={CCD}, description={\gls{Charge-Coupled Device}}}
\newglossaryentry{Center} {name={Center}, description={An entity managed by AURA that is responsible for execution of a federally funded project}}
\newglossaryentry{Charge-Coupled Device} {name={Charge-Coupled Device}, description={a particular kind of solid-state sensor for detecting optical-band photons. It is composed of a 2-D array of pixels, and one or more read-out amplifiers.}}
\newacronym{DESC} {DESC} {Dark Energy Science Collaboration}
\newacronym{DM} {DM} {\gls{Data Management}}
\newacronym{DMS} {DMS} {Data Management Subsystem}
\newglossaryentry{DMTN} {name={DMTN}, description={DM Technical Note}}
\newacronym{DTN} {DTN} {Data Transfer Node}
\newglossaryentry{Data Management} {name={Data Management}, description={The LSST Subsystem responsible for the Data Management System (DMS), which will capture, store, catalog, and serve the LSST dataset to the scientific community and public. The DM team is responsible for the DMS architecture, applications, middleware, infrastructure, algorithms, and Observatory Network Design. DM is a distributed team working at LSST and partner institutions, with the DM Subsystem Manager located at LSST headquarters in Tucson.}}
\newglossaryentry{Data Management Subsystem} {name={Data Management Subsystem}, description={The subsystems within Data Management may contain a defined combination of hardware, a software stack, a set of running processes, and the people who manage them: they are a major component of the DM System operations. Examples include the 'Archive Operations Subsystem' and the 'Data Processing Subsystem'"."}}
\newglossaryentry{Data Management System} {name={Data Management System}, description={The computing infrastructure, middleware, and applications that process, store, and enable information extraction from the LSST dataset; the DMS will process peta-scale data volume, convert raw images into a faithful representation of the universe, and archive the results in a useful form. The infrastructure layer consists of the computing, storage, networking hardware, and system software. The middleware layer handles distributed processing, data access, user interface, and system operations services. The applications layer includes the data pipelines and the science data archives' products and services.}}
\newglossaryentry{Director} {name={Director}, description={The person responsible for the overall conduct of the project; the LSST director is charged with ensuring that both the scientific goals and management constraints on the project are met. S/he is the principal public spokesperson for the project in all matters and represents the project to the scientific community, AURA, the member institutions of LSSTC, and the funding agencies.}}
\newglossaryentry{DocuShare} {name={DocuShare}, description={The trade name for the enterprise management software used by LSST to archive and manage documents}}
\newglossaryentry{Document} {name={Document}, description={Any object (in any application supported by DocuShare or design archives such as PDMWorks or GIT) that supports project management or records milestones and deliverables of the LSST Project}}
\newacronym{FIU} {FIU} {Florida International University}
\newglossaryentry{Handle} {name={Handle}, description={The unique identifier assigned to a document uploaded to DocuShare}}
\newglossaryentry{LDM} {name={LDM}, description={LSST Data Management (Document Handle)}}
\newacronym{LSST} {LSST} {Large Synoptic Survey Telescope}
\newglossaryentry{LSSTC} {name={LSSTC}, description={\gls{LSST} Corporation. an Arizona 501(c)3 not-for-profit corporation formed in 2003 for the purpose of designing, constructing, and operating the LSST System. During design and development, the Corporation stewarded private funding used for such essential contributions as early site preparation, mirror construction, and early data management system development. During construction, LSSTC will secure private operations funding from international affiliates and play a key role in preparing the scientific community to use the LSST dataset.}}
\newglossaryentry{NCSA} {name={NCSA}, description={National Center for Supercomputing Applications}}
\newglossaryentry{National Science Foundation} {name={National Science Foundation}, description={primary federal agency supporting research in all fields of fundamental science and engineering; NSF selects and funds projects through competitive, merit-based review}}
\newglossaryentry{Operations} {name={Operations}, description={The 10-year period following construction and commissioning during which the LSST Observatory conducts its survey}}
\newglossaryentry{Operations Rehersal} {name={Operations Rehersal}, description={A data management system prototype project employing the same methods, tools, personnel, and technologies as the real system in order to introduce and validate new algorithms, functionality, and infrastructure. Previously referred to as a data challenge.}}
\newacronym{PSF} {PSF} {Point Spread Function}
\newacronym{PST} {PST} {\gls{Project Science Team}}
\newglossaryentry{Project Manager} {name={Project Manager}, description={The person responsible for exercising leadership and oversight over the entire LSST project; he or she controls schedule, budget, and all contingency funds}}
\newglossaryentry{Project Science Team} {name={Project Science Team}, description={an operational unit within LSST that carries out specific scientific performance investigations as prioritized by the Director, the Project Manager, and the Project Scientist. Its membership includes key scientists on the Project who provide specific necessary expertise. The Project Science Team provides required scientific input on critical technical decisions as the project construction proceeds}}
\newglossaryentry{Project Scientist} {name={Project Scientist}, description={The principal scientific advisor to the LSST Project Manager to ensure that LSST system specifications are appropriate for achieving the scientific goals of the project; the Project Scientist also works closely with the Systems Engineering group and chairs the LSST Science Council}}
\newacronym{QA} {QA} {Quality Assurance}
\newglossaryentry{Science Collaboration} {name={Science Collaboration}, description={An autonomous body of scientists interested in a particular area of science enabled by the LSST dataset, which through precursor studies, simulations, and algorithm development lays the groundwork for the large-scale science projects the LSST will enable.  In addition to preparing their members to take full advantage of LSST early in its operations phase, the science collaborations have helped to define the system's science requirements, refine and promote the science case, and quality check design and development work.}}
\newglossaryentry{Subsystem} {name={Subsystem}, description={A set of elements comprising a system within the larger LSST system that is responsible for a key technical deliverable of the project.}}
\newglossaryentry{Subsystem Manager} {name={Subsystem Manager}, description={responsible manager for an LSST subsystem; he or she exercises authority, within prescribed limits and under scrutiny of the Project Manager, over the relevant subsystem's cost, schedule, and work plans}}
\newglossaryentry{Systems Engineering} {name={Systems Engineering}, description={an interdisciplinary field of engineering that focuses on how to design and manage complex engineering systems over their life cycles. Issues such as requirements engineering, reliability, logistics, coordination of different teams, testing and evaluation, maintainability and many other disciplines necessary for successful system development, design, implementation, and ultimate decommission become more difficult when dealing with large or complex projects. Systems engineering deals with work-processes, optimization methods, and risk management tools in such projects. It overlaps technical and human-centered disciplines such as industrial engineering, control engineering, software engineering, organizational studies, and project management. Systems engineering ensures that all likely aspects of a project or system are considered, and integrated into a whole.}}
\newacronym{US} {US} {United States}
\newglossaryentry{algorithm} {name={algorithm}, description={A computational implementation of a calculation or some method of processing.}}
\newglossaryentry{patch} {name={patch}, description={An quadrilateral sub-region of a sky tract, with a size in pixels chosen to fit easily into memory on desktop computers.}}
\newglossaryentry{sky map} {name={sky map}, description={A sky tesselization for LSST. The Stack includes software to define a geometric mapping from the representation of World Coordinates in input images to the LSST sky map. This tesselation is comprised of individual tracts which are, in turn, comprised of patches.}}
\newglossaryentry{stack} {name={stack}, description={A record of all versions of a document uploaded to a particular DocuShare handle}}
\newglossaryentry{tract} {name={tract}, description={A portion of sky, a spherical convex polygon, within the LSST all-sky tessellation (sky map). Each tract is subdivided into sky patches.}}
