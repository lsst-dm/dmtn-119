\subsection{Day 2} \label{sec:day2}

The daily meeting took place as planned at 11:00 \gls{PST}.

Transfers were initiated earlier than Day 1 and the 5 second delays removed and all data had arrived at the \gls{LDF} by 7:30 am.
Data ingestion {\texttt{/project/OpsRehearsal\_1/night2} and processing were initiated shortly thereafter and were completed within roughly 1 hour.

No processing errors were reported so examination of less severe issues were examined.
Noted were WARN-level problems that revolved around reference catalogs being in an outdated format
and the lack of zeropoint information for z- and Y-bands. 

\subsubsection{Discussion}
There were discussions about how the current pipeline, operating on DC2 data might compare with what might
be needed during commissioning and operations.  
It was noted that because the rehearsal was using a processing pipeline (and \gls{QA} tools) more amenable to Data Release Processing (\gls{DRP}) and that \gls{QA} products available might not be good comparisons to that needed when working with prompt processing.  
Also, it was noted that prompt processing \gls{QA} might form a basis for later selection of inputs to \gls{DRP}.

Another discussion revolved around whether WARN-level diagnostics might be dealt with in Operations/Commissioning.
Mostly it was felt that these fell into two classes:  those that were indicative of poor data (which might not require any intervention) and those that indicated software bugs (which should be tracked/resolved through tickets to the \gls{DM} developers).

