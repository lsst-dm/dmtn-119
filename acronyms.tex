\addtocounter{table}{-1}
\begin{longtable}{|l|p{0.8\textwidth}|}\hline
\textbf{Acronym} & \textbf{Description}  \\\hline

AD & Associate \gls{Director} \\\hline
AURA & \gls{Association of Universities for Research in Astronomy} \\\hline
Archive & The repository for documents required by the \gls{NSF} to be kept. These include documents related to design and development, construction, integration, test, and operations of the \gls{LSST} observatory system. The archive is maintained using the enterprise content management system \gls{DocuShare}, which is accessible through a link on the project website www.project.lsst.org. \\\hline
Association Pipeline & An application that matches detected Sources or DIASources or generated Objects to an existing catalog of Objects, producing a (possibly many-to-many) set of associations and a list of unassociated inputs. Association Pipelines are used in \gls{Prompt Processing} after \gls{DIASource} generation and in the final stages of \gls{Data Release} processing to ensure continuity of \gls{Object} identifiers. \\\hline
Association of Universities for Research in Astronomy &  consortium of \gls{US} institutions and international affiliates that operates world-class astronomical observatories, \gls{AURA} is the legal entity responsible for managing what it calls independent operating Centers, including \gls{LSST}, under respective cooperative agreements with the \gls{National Science Foundation}. \gls{AURA} assumes fiducial responsibility for the funds provided through those cooperative agreements. \gls{AURA} also is the legal owner of the \gls{AURA} Observatory properties in Chile. \\\hline
CCD & \gls{Charge-Coupled Device} \\\hline
Center & An entity managed by \gls{AURA} that is responsible for execution of a federally funded project \\\hline
Charge-Coupled Device & a particular kind of solid-state sensor for detecting optical-band photons. It is composed of a 2-D array of pixels, and one or more read-out amplifiers. \\\hline
Commissioning & A two-year phase at the end of the \gls{Construction} project during which a technical team a) integrates the various technical components of the three subsystems; b) shows their compliance with ICDs and system-level requirements as detailed in the \gls{LSST} Observatory System Specifications document (\gls{OSS}, \gls{LSE}-30); and c) performs science verification to show compliance with the survey performance specifications as detailed in the \gls{LSST} Science Requirements \gls{Document} (\gls{SRD}, \gls{LPM}-17). \\\hline
Construction & The period during which \gls{LSST} observatory facilities, components, hardware, and software are built, tested, integrated, and commissioned. \gls{Construction} follows design and development and precedes operations. The \gls{LSST} construction phase is funded through the \gls{NSF} \gls{MREFC} account. \\\hline
DESC & Dark Energy \gls{Science Collaboration} \\\hline
DM & \gls{Data Management} \\\hline
DMS & \gls{Data Management Subsystem} \\\hline
DMTN & \gls{DM} Technical Note \\\hline
DR & \gls{Data Release} \\\hline
DRP & \gls{Data Release Production} \\\hline
DTN & Data Transfer Node \\\hline
Data Management & The \gls{LSST} Subsystem responsible for the \gls{Data Management System} (\gls{DMS}), which will capture, store, catalog, and serve the \gls{LSST} dataset to the scientific community and public. The \gls{DM} team is responsible for the \gls{DMS} architecture, applications, middleware, infrastructure, algorithms, and Observatory Network Design. \gls{DM} is a distributed team working at \gls{LSST} and partner institutions, with the \gls{DM} \gls{Subsystem Manager} located at \gls{LSST} headquarters in Tucson. \\\hline
Data Management \gls{Subsystem} & The subsystems within \gls{Data Management} may contain a defined combination of hardware, a software \gls{stack}, a set of running processes, and the people who manage them: they are a major component of the \gls{DM} System operations. Examples include the 'Archive \gls{Operations} \gls{Subsystem}' and the 'Data Processing \gls{Subsystem}'"." \\\hline
Data Release & The approximately annual reprocessing of all \gls{LSST} data, and the installation of the resulting data products in the \gls{LSST} Data Access Centers, which marks the start of the two-year proprietary period. \\\hline
Data Release Processing & Deprecated term; see \gls{Data Release Production}. \\\hline
Data Release Production & An episode of (re)processing all of the accumulated \gls{LSST} images, during which all output \gls{DR} data products are generated. These episodes are planned to occur annually during the \gls{LSST} survey, and the processing will be executed at the \gls{Archive} \gls{Center}. This includes Difference Imaging Analysis, generating deep Coadd Images, \gls{Source} detection and association, creating Object and \gls{Solar System Object} catalogs, and related \gls{metadata}. \\\hline
Director & The person responsible for the overall conduct of the project; the \gls{LSST} director is charged with ensuring that both the scientific goals and management constraints on the project are met. S/he is the principal public spokesperson for the project in all matters and represents the project to the scientific community, \gls{AURA}, the member institutions of \gls{LSSTC}, and the funding agencies. \\\hline
DocuShare & The trade name for the enterprise management software used by \gls{LSST} to archive and manage documents \\\hline
Document & Any object (in any application supported by \gls{DocuShare} or design archives such as PDMWorks or GIT) that supports project management or records milestones and deliverables of the \gls{LSST} Project \\\hline
FITS & \gls{Flexible Image Transport System} \\\hline
FIU & Florida International University \\\hline
Flexible Image Transport System & an international standard in astronomy for storing images, tables, and \gls{metadata} in disk files. See the \gls{IAU} \gls{FITS} Standard for details. \\\hline
Handle & The unique identifier assigned to a document uploaded to \gls{DocuShare} \\\hline
IAU & International Astronomical Union \\\hline
LDF & \gls{LSST} Data Facility \\\hline
LDM & \gls{LSST} \gls{Data Management} (\gls{Document} \gls{Handle}) \\\hline
LPM & \gls{LSST} Project Management (\gls{Document} \gls{Handle}) \\\hline
LSE & \gls{LSST} \gls{Systems Engineering} (\gls{Document} \gls{Handle}) \\\hline
LSST & Large Synoptic Survey Telescope \\\hline
LSSTC & \gls{LSST} Corporation. an Arizona 501(c)3 not-for-profit corporation formed in 2003 for the purpose of designing, constructing, and operating the \gls{LSST} System. During design and development, the Corporation stewarded private funding used for such essential contributions as early site preparation, mirror construction, and early data management system development. During construction, \gls{LSSTC} will secure private operations funding from international affiliates and play a key role in preparing the scientific community to use the \gls{LSST} dataset. \\\hline
MOPS & \gls{Moving Object Processing System} \\\hline
MREFC & Major Research Equipment and Facility \gls{Construction} \\\hline
Major Research Equipment and Facility \gls{Construction} & the \gls{NSF} account through which large facilities construction projects such as \gls{LSST} are funded \\\hline
Moving Object Processing System & The \gls{Moving Object Processing System} (\gls{MOPS}) identifies new SSObjects using unassociated DIASources. \gls{MOPS} is part of the \gls{Science Pipelines}. \\\hline
NCSA & National \gls{Center} for Supercomputing Applications \\\hline
NSF & \gls{National Science Foundation} \\\hline
National Science Foundation & primary federal agency supporting research in all fields of fundamental science and engineering; \gls{NSF} selects and funds projects through competitive, merit-based review \\\hline
OSS & Observatory System Specifications; \gls{LSE}-30 \\\hline
Object & In \gls{LSST} nomenclature this refers to an \gls{astronomical object}, such as a star, galaxy, or other physical entity. E.g., comets, asteroids are also Objects but typically called a Moving Object or a \gls{Solar System Object} (SSObject). One of the \gls{DRP} data products is a table of Objects detected by \gls{LSST} which can be static, or change brightness or position with time. \\\hline
Operations & The 10-year period following construction and commissioning during which the \gls{LSST} Observatory conducts its survey \\\hline
PSF & Point Spread Function \\\hline
PST & \gls{Project Science Team} \\\hline
Project Manager & The person responsible for exercising leadership and oversight over the entire \gls{LSST} project; he or she controls schedule, budget, and all contingency funds \\\hline
Project Science Team & an operational unit within \gls{LSST} that carries out specific scientific performance investigations as prioritized by the \gls{Director}, the \gls{Project Manager}, and the \gls{Project Scientist}. Its membership includes key scientists on the Project who provide specific necessary expertise. The \gls{Project Science Team} provides required scientific input on critical technical decisions as the project construction proceeds \\\hline
Project Scientist & The principal scientific advisor  to the \gls{LSST} \gls{Project Manager} to ensure that \gls{LSST} system specifications are appropriate for achieving the scientific goals of the project; the \gls{Project Scientist} also works closely with the \gls{Systems Engineering} group and chairs the \gls{LSST} Science Council \\\hline
QA & \gls{Quality Assurance} \\\hline
QC & \gls{Quality Control} \\\hline
Quality Assurance & All activities, deliverables, services, documents, procedures or artifacts which are designed to ensure the quality of DM deliverables. This may include \gls{QC} systems, in so far as they are covered in the charge described in \gls{LDM}-622. Note that contrasts with the \gls{LDM}-522 definition of “QA” as “Quality Analysis”, a manual process which occurs only during commissioning and operations. See also: \gls{Quality Control}. \\\hline
Quality Control & Services and processes which are aimed at measuring and \gls{monitoring} a system to verify and characterize its performance (as in \gls{LDM}-522). \gls{Quality Control} systems run autonomously, only notifying people when an anomaly has been detected. See also \gls{Quality Assurance}. \\\hline
SRD & \gls{LSST} Science Requirements; \gls{LPM}-17 \\\hline
Science Collaboration & An autonomous body of scientists interested in a particular area of science enabled by the \gls{LSST} dataset, which through precursor studies, simulations, and \gls{algorithm} development lays the groundwork for the large-scale science projects the \gls{LSST} will enable.  In addition to preparing their members to take full advantage of \gls{LSST} early in its operations phase, the science collaborations have helped to define the system's science requirements, refine and promote the science case, and quality check design and development work. \\\hline
Science Pipelines & The library of software components and the algorithms and processing pipelines assembled from them that are being developed by \gls{DM} to generate science-ready data products from \gls{LSST} images. The Pipelines may be executed at scale as part of \gls{LSST} Prompt or \gls{Data Release} processing, or pieces of them may be used in a standalone mode or executed through the \gls{LSST} \gls{Science Platform}. The \gls{Science Pipelines} are one component of the \gls{LSST} \gls{Software Stack}. \\\hline
Solar System \gls{Object} & A solar system object is an astrophysical object that is identified as part of the Solar System: planets and their satellites, asteroids, comets, etc. This class of object had historically been referred to within the \gls{LSST} Project as Moving Objects. \\\hline
Source & A single detection of an astrophysical object in an image, the characteristics for which are stored in the \gls{Source} Catalog of the \gls{DRP} database. The association of Sources that are non-moving lead to Objects; the association of moving Sources leads to Solar System Objects. (Note that in non-LSST usage "source" is often used for what \gls{LSST} calls an \gls{Object}.) \\\hline
Subsystem & A set of elements comprising a system within the larger \gls{LSST} system that is responsible for a key technical deliverable of the project. \\\hline
Subsystem Manager & responsible manager for an LSST subsystem; he or she exercises authority, within prescribed limits and under scrutiny of the Project Manager, over the relevant subsystem's cost, schedule, and work plans \\\hline
Systems Engineering & an interdisciplinary field of engineering that focuses on how to design and manage complex engineering systems over their life cycles. Issues such as requirements engineering, reliability, logistics, coordination of different teams, testing and evaluation, maintainability and many other disciplines necessary for successful system development, design, implementation, and ultimate decommission become more difficult when dealing with large or complex projects. Systems engineering deals with work-processes, optimization methods, and risk management tools in such projects. It overlaps technical and human-centered disciplines such as industrial engineering, control engineering, software engineering, organizational studies, and project management. Systems engineering ensures that all likely aspects of a project or system are considered, and integrated into a whole. \\\hline
US & United States \\\hline
Visit & A sequence of one or more consecutive exposures at a given position, orientation, and filter within the \gls{LSST} cadence. See Standard Visit, \gls{Alternative Standard Visit}, and \gls{Non-Standard Visit},\gls{DM} \gls{TS} Sims,,
Education and Public Outreach (\gls{EPO}),The \gls{LSST} subsystem responsible for the cyberinfrastructure \\\hline
algorithm & A computational implementation of a calculation or some method of processing. \\\hline
metadata & General term for data about data, e.g., attributes of astronomical objects (e.g. images, sources, astroObjects, etc.) that are characteristics of the objects themselves, and facilitate the organization, preservation, and query of data sets. (E.g., a \gls{FITS} header contains \gls{metadata}). \\\hline
patch & An quadrilateral sub-region of a sky \gls{tract}, with a size in pixels chosen to fit easily into memory on desktop computers. \\\hline
pipeline & A configured sequence of software tasks (Stages) to process data and generate data products. Example: \gls{Association Pipeline}. \\\hline
sky map & A sky tessellation for \gls{LSST}. The Stack includes software to define a geometric mapping from the representation of World Coordinates in input images to the \gls{LSST} \gls{sky map}. This tessellation is comprised of individual tracts which are, in turn, comprised of patches. \\\hline
stack & a grouping, usually in layers (hence \gls{stack}), of software packages and services to achieve a common goal. Often providing a higher level set of end user oriented services and tools \\\hline
tract & A portion of sky, a spherical convex polygon, within the \gls{LSST} all-sky tessellation (\gls{sky map}). Each \gls{tract} is subdivided into sky patches. \\\hline
\end{longtable}
