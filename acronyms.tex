\addtocounter{table}{-1}
\begin{longtable}{|l|p{0.8\textwidth}|}\hline
\textbf{Acronym} & \textbf{Description}  \\\hline

AD & Associate Director \\\hline
AURA & \gls{Association of Universities for Research in Astronomy} \\\hline
Archive Center & Part of the LSST Data Management System, the LSST archive center is a data center at NCSA that hosts the LSST Archive, which includes released science data and metadata, observatory and engineering data, and supporting software such as the LSST Software Stack. \\\hline
Association Pipeline & An application that matches detected Sources or DIASources or generated Objects to an existing catalog of Objects, producing a (possibly many-to-many) set of associations and a list of unassociated inputs. Association Pipelines are used in Prompt Processing after DIASource generation and in the final stages of Data Release processing to ensure continuity of Object identifiers. \\\hline
CCD & \gls{Charge-Coupled Device} \\\hline
Center & An entity managed by AURA that is responsible for execution of a federally funded project \\\hline
Charge-Coupled Device & a particular kind of solid-state sensor for detecting optical-band photons. It is composed of a 2-D array of pixels, and one or more read-out amplifiers. \\\hline
Commissioning & A two-year phase at the end of the Construction project during which a technical team a) integrates the various technical components of the three subsystems; b) shows their compliance with ICDs and system-level requirements as detailed in the LSST Observatory System Specifications document (OSS, LSE-30); and c) performs science verification to show compliance with the survey performance specifications as detailed in the LSST Science Requirements Document (SRD, LPM-17). \\\hline
Construction & The period during which LSST observatory facilities, components, hardware, and software are built, tested, integrated, and commissioned. Construction follows design and development and precedes operations. The LSST construction phase is funded through the \gls{NSF} \gls{MREFC} account. \\\hline
DESC & Dark Energy Science Collaboration \\\hline
DM & \gls{Data Management} \\\hline
DMS & Data Management Subsystem \\\hline
DMTN & DM Technical Note \\\hline
DR & Data Release \\\hline
DRP & Data Release Production \\\hline
DTN & Data Transfer Node \\\hline
Data Management & The LSST Subsystem responsible for the Data Management System (DMS), which will capture, store, catalog, and serve the LSST dataset to the scientific community and public. The DM team is responsible for the DMS architecture, applications, middleware, infrastructure, algorithms, and Observatory Network Design. DM is a distributed team working at LSST and partner institutions, with the DM Subsystem Manager located at LSST headquarters in Tucson. \\\hline
Data Management System & The computing infrastructure, middleware, and applications that process, store, and enable information extraction from the LSST dataset; the DMS will process peta-scale data volume, convert raw images into a faithful representation of the universe, and archive the results in a useful form. The infrastructure layer consists of the computing, storage, networking hardware, and system software. The middleware layer handles distributed processing, data access, user interface, and system operations services. The applications layer includes the data pipelines and the science data archives' products and services. \\\hline
Data Release Processing & Deprecated term; see Data Release Production. \\\hline
Data Release Production & An episode of (re)processing all of the accumulated LSST images, during which all output DR data products are generated. These episodes are planned to occur annually during the LSST survey, and the processing will be executed at the Archive Center. This includes Difference Imaging Analysis, generating deep Coadd Images, Source detection and association, creating Object and Solar System Object catalogs, and related metadata. \\\hline
Director & The person responsible for the overall conduct of the project; the LSST director is charged with ensuring that both the scientific goals and management constraints on the project are met. S/he is the principal public spokesperson for the project in all matters and represents the project to the scientific community, AURA, the member institutions of LSSTC, and the funding agencies. \\\hline
DocuShare & The trade name for the enterprise management software used by LSST to archive and manage documents \\\hline
Document & Any object (in any application supported by DocuShare or design archives such as PDMWorks or GIT) that supports project management or records milestones and deliverables of the LSST Project \\\hline
FIU & Florida International University \\\hline
Handle & The unique identifier assigned to a document uploaded to DocuShare \\\hline
LDF & LSST Data Facility \\\hline
LDM & LSST Data Management (Document Handle) \\\hline
LPM & LSST Project Management (Document Handle) \\\hline
LSE & LSST Systems Engineering (Document Handle) \\\hline
LSST & Large Synoptic Survey Telescope \\\hline
LSSTC & \gls{LSST} Corporation. an Arizona 501(c)3 not-for-profit corporation formed in 2003 for the purpose of designing, constructing, and operating the LSST System. During design and development, the Corporation stewarded private funding used for such essential contributions as early site preparation, mirror construction, and early data management system development. During construction, LSSTC will secure private operations funding from international affiliates and play a key role in preparing the scientific community to use the LSST dataset. \\\hline
NCSA & National Center for Supercomputing Applications \\\hline
OSS & Observatory System Specifications; LSE-30 \\\hline
Object & In LSST nomenclature this refers to an astronomical object, such as a star, galaxy, or other physical entity. E.g., comets, asteroids are also Objects but typically called a Moving Object or a Solar System Object (SSObject). One of the DRP data products is a table of Objects detected by LSST which can be static, or change brightness or position with time. \\\hline
Operations & The 10-year period following construction and commissioning during which the LSST Observatory conducts its survey \\\hline
Operations Rehersal & A data management system prototype project employing the same methods, tools, personnel, and technologies as the real system in order to introduce and validate new algorithms, functionality, and infrastructure. Previously referred to as a data challenge. \\\hline
PSF & Point Spread Function \\\hline
PST & \gls{Project Science Team} \\\hline
Project Manager & The person responsible for exercising leadership and oversight over the entire LSST project; he or she controls schedule, budget, and all contingency funds \\\hline
Project Science Team & an operational unit within LSST that carries out specific scientific performance investigations as prioritized by the Director, the Project Manager, and the Project Scientist. Its membership includes key scientists on the Project who provide specific necessary expertise. The Project Science Team provides required scientific input on critical technical decisions as the project construction proceeds \\\hline
Project Scientist & The principal scientific advisor to the LSST Project Manager to ensure that LSST system specifications are appropriate for achieving the scientific goals of the project; the Project Scientist also works closely with the Systems Engineering group and chairs the LSST Science Council \\\hline
QA & Quality Assurance \\\hline
SRD & LSST Science Requirements; LPM-17 \\\hline
Science Collaboration & An autonomous body of scientists interested in a particular area of science enabled by the LSST dataset, which through precursor studies, simulations, and algorithm development lays the groundwork for the large-scale science projects the LSST will enable.  In addition to preparing their members to take full advantage of LSST early in its operations phase, the science collaborations have helped to define the system's science requirements, refine and promote the science case, and quality check design and development work. \\\hline
Solar System Object & A solar system object is an astrophysical object that is identified as part of the Solar System: planets and their satellites, asteroids, comets, etc. This class of object had historically been referred to within the LSST Project as Moving Objects. \\\hline
Source & A single detection of an astrophysical object in an image, the characteristics for which are stored in the Source Catalog of the DRP database. The association of Sources that are non-moving lead to Objects; the association of moving Sources leads to Solar System Objects. (Note that in non-LSST usage "source" is often used for what LSST calls an Object.) \\\hline
Subsystem & A set of elements comprising a system within the larger LSST system that is responsible for a key technical deliverable of the project. \\\hline
Subsystem Manager & responsible manager for an LSST subsystem; he or she exercises authority, within prescribed limits and under scrutiny of the Project Manager, over the relevant subsystem's cost, schedule, and work plans \\\hline
algorithm & A computational implementation of a calculation or some method of processing. \\\hline
metadata & General term for data about data, e.g., attributes of astronomical objects (e.g. images, sources, astroObjects, etc.) that are characteristics of the objects themselves, and facilitate the organization, preservation, and query of data sets. (E.g., a FITS header contains metadata). \\\hline
patch & An quadrilateral sub-region of a sky tract, with a size in pixels chosen to fit easily into memory on desktop computers. \\\hline
pipeline & A configured sequence of software tasks (Stages) to process data and generate data products. Example: Association Pipeline. \\\hline
sky map & A sky tesselization for LSST. The Stack includes software to define a geometric mapping from the representation of World Coordinates in input images to the LSST sky map. This tesselation is comprised of individual tracts which are, in turn, comprised of patches. \\\hline
tract & A portion of sky, a spherical convex polygon, within the LSST all-sky tessellation (sky map). Each tract is subdivided into sky patches. \\\hline
\end{longtable}
