\documentclass[DM,authoryear,toc]{lsstdoc}
% lsstdoc documentation: https://lsst-texmf.lsst.io/lsstdoc.html

\input{meta}

% Package imports go here.

% Local commands go here.

% To add a short-form title:
% \title[Short title]{Title}
\title{Report on Operations Rehersal \#1}

% Optional subtitle
% \setDocSubtitle{A subtitle}

\author{%
William O'Mullane
}

\setDocRef{DMTN-119}
\setDocUpstreamLocation{\url{https://github.com/lsst-dm/dmtn-119}}

\date{\vcsDate}

% Optional: name of the document's curator
% \setDocCurator{The Curator of this Document}

\setDocAbstract{%
Short report on the ops rehearsal held in Feb 2019.
}

% Change history defined here.
% Order: oldest first.
% Fields: VERSION, DATE, DESCRIPTION, OWNER NAME.
% See LPM-51 for version number policy.
\setDocChangeRecord{%
  \addtohist{1}{YYYY-MM-DD}{Unreleased.}{William O'Mullane}
}

\begin{document}

% Create the title page.
\maketitle

% ADD CONTENT HERE
% You can also use the \input command to include several content files.

\appendix
% Include all the relevant bib files.
% https://lsst-texmf.lsst.io/lsstdoc.html#bibliographies
\section{References} \label{sec:bib}
\bibliography{local,lsst,lsst-dm,refs_ads,refs,books}

% Make sure lsst-texmf/bin/generateAcronyms.py is in your path
\section{Acronyms} \label{sec:acronyms}
\addtocounter{table}{-1}
\begin{longtable}{|l|p{0.8\textwidth}|}\hline
\textbf{Acronym} & \textbf{Description}  \\\hline

AD & Associate Director \\\hline
CCD & Charge-Coupled Device \\\hline
DESC & Dark Energy Science Collaboration \\\hline
DM & Data Management \\\hline
DMTN & \gls{DM} Technical Note \\\hline
DRP & Data Release Production \\\hline
DTN & Data Transfer Node \\\hline
FIU & Florida International University \\\hline
LDF & \gls{LSST} Data Facility \\\hline
LDM & \gls{LSST} Data Management (Document Handle) \\\hline
LSST & Large Synoptic Survey Telescope \\\hline
NCSA & National Center for Supercomputing Applications \\\hline
PSF & Point Spread Function \\\hline
PST & Project Science Team \\\hline
QA & Quality Assurance \\\hline
\end{longtable}


\end{document}
